\batchmode
% THIS IS SIGPROC-SP.TEX - VERSION 3.1
% WORKS WITH V3.2SP OF ACM_PROC_ARTICLE-SP.CLS
% APRIL 2009
%
% It is an example file showing how to use the 'acm_proc_article-sp.cls' V3.2SP
% LaTeX2e document class file for Conference Proceedings submissions.
% ----------------------------------------------------------------------------------------------------------------
% This .tex file (and associated .cls V3.2SP) *DOES NOT* produce:
%       1) The Permission Statement
%       2) The Conference (location) Info information
%       3) The Copyright Line with ACM data
%       4) Page numbering
% ---------------------------------------------------------------------------------------------------------------
% It is an example which *does* use the .bib file (from which the .bbl file
% is produced).
% REMEMBER HOWEVER: After having produced the .bbl file,
% and prior to final submission,
% you need to 'insert'  your .bbl file into your source .tex file so as to provide
% ONE 'self-contained' source file.
%
% Questions regarding SIGS should be sent to
% Adrienne Griscti ---> griscti@acm.org
%
% Questions/suggestions regarding the guidelines, .tex and .cls files, etc. to
% Gerald Murray ---> murray@hq.acm.org
%
% For tracking purposes - this is V3.1SP - APRIL 2009

\documentclass{sig-alternate}
\usepackage{float}
\usepackage{Sweavel}
\usepackage{placeins}
\usepackage{comment}
\usepackage[bookmarks=true, urlcolor=blue, linkcolor=blue, citecolor=blue, colorlinks=true, pdftitle={MyTitle}, pdfauthor={Me}]{hyperref}
\usepackage[center]{caption}

\usepackage[font=footnotesize]{subfig}

\DeclareCaptionType{copyrightbox}
\setcounter{topnumber}{2}
\setcounter{bottomnumber}{2}
\setcounter{totalnumber}{4}
\renewcommand{\topfraction}{0.85}
\renewcommand{\bottomfraction}{0.85}
\renewcommand{\textfraction}{0.15}
\renewcommand{\floatpagefraction}{0.7}

\newcommand{\ignore}[1]{}
 %------------begin Float Adjustment
%two column float page must be 90% full
%\renewcommand\dblfloatpagefraction{.90}
%two column top float can cover up to 80% of page
%\renewcommand\dbltopfraction{.80}
%float page must be 90% full
%\renewcommand\floatpagefraction{.90}
%top float can cover up to 80% of page
%\renewcommand\topfraction{.80}
%bottom float can cover up to 80% of page
%\renewcommand\bottomfraction{.80}
%at least 10% of a normal page must contain text
%\renewcommand\textfraction{.1}
%separation between floats and text
\setlength\dbltextfloatsep{9pt plus 5pt minus 3pt }
%separation between two column floats and text
\setlength\textfloatsep{4pt plus 2pt minus 1.5pt}

\begin{document}

\title{My Title}
\subtitle{My subtitle}


%
% You need the command \numberofauthors to handle the 'placement
% and alignment' of the authors beneath the title.
%
% For aesthetic reasons, we recommend 'three authors at a time'
% i.e. three 'name/affiliation blocks' be placed beneath the title.
%
% NOTE: You are NOT restricted in how many 'rows' of
% "name/affiliations" may appear. We just ask that you restrict
% the number of 'columns' to three.
%
% Because of the available 'opening page real-estate'
% we ask you to refrain from putting more than six authors
% (two rows with three columns) beneath the article title.
% More than six makes the first-page appear very cluttered indeed.
%
% Use the \alignauthor commands to handle the names
% and affiliations for an 'aesthetic maximum' of six authors.
% Add names, affiliations, addresses for
% the seventh etc. author(s) as the argument for the
% \additionalauthors command.
% These 'additional authors' will be output/set for you
% without further effort on your part as the last section in
% the body of your article BEFORE References or any Appendices.

\author{
Author 1\\
       \affaddr{Affiliate}\\
       \email{a1@affiliate}
\alignauthor
Author 2\\
       \affaddr{Affiliate}\\
       \email{a2@affiliate}
}



\maketitle
\begin{abstract}
% 1.State the problem
%2.Say why it’s an interestin gproblem
%3.Say what your solution achieves
%4.Say what follows from your solution
My Abstract

\end{abstract}

%This is from the 98 class, 2012 does not have the number.
\category{D.2.5}{Software Engineering}{ACM Category}{ Category}

\terms{My, Terms}

\keywords{my,key,words} % NOT required for Proceedings


\input{paper}
%
% The following two commands are all you need in the
% initial runs of your .tex file to
% produce the bibliography for the citations in your paper.
\bibliographystyle{abbrv}
\bibliography{paper}  % sigproc.bib is the name of the Bibliography in this case
\end{document}
